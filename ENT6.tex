\documentclass{article} 

% the purpose of this particular document is to serve as a staging area potential blog posts, for the purpose of editing, revision, and review. 

% packages 
	\usepackage{amsmath, amsthm, mathtools}
	\usepackage{mdframed}  
	\usepackage{geometry, enumerate} 
	\usepackage{parskip}
	\usepackage{graphicx}
	\usepackage{dsfont} % for math bf text 

% theorems and such 
  	\newtheorem{theorem}{Theorem}
  	\newtheorem{corollary}{Corollary}
  	\newtheorem{lemma}[theorem]{Lemma} 
  	\newtheorem*{remark}{Remark}
  	\newtheorem*{exe}{Exercise}
  	\newtheorem{prop}{Proposition}
  	\newtheorem{example}{Example} 
  	\newtheorem{definition}{Definition}   
  	
% custom commands 
	\newcommand{\ceil}[1]{\left \lceil #1 \right \rceil}
	\newcommand{\floor}[1]{\lfloor #1 \rfloor}
	\newcommand{\X}[1]{\, \text{mod} \, #1}
	\newcommand{\divv}{\,|\,}
	\newcommand{\ndiv}{\,\not|\,} 
	\newcommand{\GCD}[2]{GCD\,(#1, #2)}
	\newcommand{\LCM}[2]{LCM\,(#1, #2)}

\begin{document} 

\title{Elementary Number Theory : \S6 } 
\author{Henry Slayer $|$ University of California, Santa Cruz} 
\date{}
\maketitle

\section*{The Chinese Remainder Theorem} 
Often, we will be faced with problems or situations in which it is helpful to break congruence down into a system of congruences. Conversely, it can be useful to assemble a single congruence that uniquely describes a system of smaller congruences. Our tool for doing so is the Chinese Remainder Theorem, which allows us to relate congruence across different modulii. Speaking loosely, if our modulus $m$ can be decomposed into coprime factors, say $d$ and $e$, every congruence modulo $m$ is equivalent to a compound congruence $[a, b]\X{[d, e]}$. The bracket notation means \textit{congruent to \textbf{a} modulo \textbf{d}, and \textbf{b} modulo \textbf{e}}. More formally, we can say the following 
\begin{mdframed} 
\begin{theorem}[Chinese Remainder Theorem] 
If $d$ and $e$ are a pair of coprime numbers, then there exists a bijective correspondence between the set of pairs $[a, b]$, where $0\leq a < d$, and $0\leq b < e$, and the set of integers $n$ for which $0\leq n < de$. \\
In other words, the pairs of residues $[a, b]\X{[d, e]}$ map uniquely to the residues $n\X{(de)}$. 
\end{theorem} 
\begin{proof} 
The sets that we're working with are the same size. So, if we can find an injective map from the set of pairs $[a, b]$ to the set of integers from 0 to $de - 1$, we've found our bijection. So how do we map $n\mapsto [a, b]$? There's a natural choice. \\
Let $n\mapsto [a, b]$ be defined by the rule that $n\mapsto [a, b]$ if and only if $n\equiv[a, b]\X{de}$. \\
We need to show that this map is injective, so suppose that we have a pair of values $0\leq m < de$ and $0\leq n < de$, such that $n\equiv[a, b]\X{de}$ and $m\equiv[a, b]\X{[d, e]}$. Naturally, then, $n - m \equiv [0, 0]\X{[d, e]} $. In turn, this implies that $n - m$ is a multiple of both $d$ and $e$. As $d$ and $e$ are coprime, this forces that $n - m$ is a multiple of the least common multiple of $d$ and $e$. In this case, this means that $n - m$ is a multiple of $de$. Yet, both $m$ and $n$ are strictly less than $de$, so their difference $n - m$ is surely less than $de$. The only multiple of $de$ less than $de$ is 0, so $m - n = 0$, and $m = n$. \\
Having shown that the map is injective, the fact that our two sets (the set of pairs, and the set of residue classes modulo $de$) are the same size implies that the map is bijective. \\
As the map is bijective, this implies that every congruence modulo $de$ corresponds to a unique congruence $[a, b]\X{[d, e]}$, \textbf{provided that d and e are coprime}.  
\end{proof} 
\end{mdframed}  

\subsubsection*{Disassembling and Reassembling Congruence} 
The extreme utility of the Chinese Remainder Theorem is that it enables us to reduce complex congruences to systems of simpler congruences, and vice versa. 
\begin{mdframed} 
\begin{example} 
Break $x\equiv 45\X{56}$ into a system of simpler congruences. \\\\
Breaking down congruence couldn't be simpler. All we need to do is find two (or more) coprime factors of 56. 7 and 8 seem like a good choice. $45\equiv 3\X{7}$, and $45\equiv5\X{8}$, so $45\equiv[3, 5]\X{[7, 8]}$. Brilliant. 
\end{example} 
\end{mdframed} 
\begin{mdframed}
\begin{example} 
What is $x$, if... \\
x is no more than 100. 
Counting by 3, we have one left over. \\
Counting by 22, we have three left over. \\
Counting by 7, we have 4 left over.\\\\
We'll start the problem, and leave the reader to finish it. \\
The statement of the problem implies that $x\equiv 1\X{3}$, $3\X{22}$, and $4\X{7}$. Each one of these linear congruences is really hiding a linear diophantine equation, which gives us the following system of equations: 
\begin{align*} 
x - 3y &= 1 \\
x - 22z &= 3 \\
x - 7w &= 4
\end{align*} 
You do the rest! Hint: use the first two equations to come up with a congruence modulo 66. This can be translated back into the language of linear diophantine equations, and re-combined with the last equation to construct a congruence in terms of 462. 
\end{example} 
\end{mdframed} 
\begin{mdframed} 
\begin{example} 
Find all solutions to $x^2\equiv 1\X{21}$. \\\\
By the CRT, we can decompose this into two congruences, $x^2\equiv 1\X{7}$ and $x^2\equiv1\X{3}$. Modulo 7, this means that we have two solutions: $x\equiv 6$, or $x\equiv 1$. Modulo 3, we also have two solutions, $x\equiv 1$ or $x\equiv 2$. Two options for each smaller congruence gives us four possible combinations: 
\begin{align*} 
x\equiv [1, 1]\X{[3, 7]}\\
x\equiv [1, 6]\X{[3, 7]}\\
x\equiv [2, 1]\X{[3, 7]}\\
x\equiv [2, 6]\X{[3, 7]}
\end{align*} 
Respectively, these simpler congruences resolve to $x\equiv 1,\, 8,\, 13,\, 20\X{21}$, which characterizes all solutions to $x^2\equiv 1\X{21}$.  
\end{example} 
\end{mdframed} 
The last example here characterizes a useful generalization. Solutions to the congruence $x^2\equiv a\X{de}$ are in one-to-one correspondence with the pairs of solutions $(u, v)$ to the congruences $u^2\equiv a\X{d}$ and $v^2\equiv a\X{e}$. In other words, $a$ is a square mod $de$ (meaning $a\equiv x^2$ for some $x\X{de}$) if and only if $a$ is a square modulo $d$ and modulo $e$. We'll dive into squares (the \textit{quadratic residues}) a bit more further down the road. 




\end{document} 